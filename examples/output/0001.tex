\documentclass[addpoints,12pt]{exam}
\usepackage[left=1cm,top=1cm,right=1cm,bottom=2cm,nohead,nofoot]{geometry}
\usepackage[utf8]{inputenc} 
\usepackage[czech]{babel}
\usepackage{amsmath}
\usepackage{amssymb}
\usepackage{stmaryrd}
\begin{document}
\setlength{\parindent}{0in}
\pagestyle{empty}
\textbf{ID: 0001}
\\
\textbf{2. Opravný termín, semestrální test ISJ, 8.6.2012, Login:}
\\
\hrulefill
\begin{questions}

\question[2]
Který z následujících jazyků nemá vestavěnou podporu pro regulární výrazy?
\nopagebreak
\begin{choices}
\choice
Python 3
\choice
Ruby
\choice
Python 2
\choice
C
\end{choices}
\question[2]
Který z těchto jazyků je nejvíce objektový?
\nopagebreak
\begin{choices}
\choice
C
\choice
Ruby
\choice
Python 2
\choice
Python 3
\end{choices}
\question[2]
Jaký je rozdíl mezi seznamem (list) a n-ticí (tuple) v Pythonu?
\nopagebreak
\begin{choices}
\choice
Pro n-tici platí, že nelze měnit její velikost ani obsah.
\choice
N-tici nelze iterovat.
\choice
Na n-tici nelze aplikovat operátor *.
\choice
Seznam není vestavěný typ.
\end{choices}
\question[2]
K čemu slouží pass příkaz v Pythonu?
\nopagebreak
\begin{choices}
\choice
Prázdná operace.
\choice
Vyhození výjimky.
\choice
Ukončení programu s návratovou hodnotou 0.
\choice
Násilné ukončení cyklu.
\end{choices}
\question[2]
K čemu slouží příkaz puts v jazyk Ruby?
\nopagebreak
\begin{choices}
\choice
Zvýšení velikosti pole o 1.
\choice
Přidání prvku do pole.
\choice
Tisk řetězce na výstup.
\choice
Vyjmutí prvního prvku z pole.
\end{choices}
\question[2]
Která z následujících tvrzení platí pro datový typ seznam (list) v Pythonu?
\nopagebreak
\begin{choices}
\choice
Nepatří do vestavěných typů jazyka Python.
\choice
Je takzvaným immutable typem.
\choice
Můžeme měnit jeho velikost.
\choice
Můžeme měnit jeho obsah.
\end{choices}
\question[2]
Které řetězce odpovídají následujícímu regulárnímu výrazu [0-9a-fA-F]\{2\}?
\nopagebreak
\begin{choices}
\choice
16.0
\choice
a9
\choice
0L
\choice
147
\end{choices}
\question[2]
Které z následujících jazyků mají zabudovanou podporu (standardní knihovnu) regulárních výrazů?
\nopagebreak
\begin{choices}
\choice
Python
\choice
Ruby
\choice
C
\choice
Java SE
\end{choices}
\question[2]
Který z následujících příkazů neexistuje v jazyce Python?
\nopagebreak
\begin{choices}
\choice
cut
\choice
puts
\choice
del
\choice
raise
\end{choices}
\question[2]
Která z následujících tvrzení platí pro datový typ n-tice (tuple) v jazyce Python?
\nopagebreak
\begin{choices}
\choice
Nepatří do vestavěných typů jazyka Python.
\choice
Je takzvaným immutable typem.
\choice
Můžeme měnit jeho obsah.
\choice
Můžeme měnit jeho velikost.
\end{choices}
\question[2]
Napište vše co víte o iterátorech v Pythonu.
\nopagebreak
\makeemptybox{10cm}
\question[2]
Napište jediným RE přidání oddělení řádu mocnin tečkou (1214248 -\textgreater{} 1.214.248).
\nopagebreak
\makeemptybox{10cm}
\question[2]
Srovnejte jazyky Ruby a Python.
\nopagebreak
\makeemptybox{10cm}
\question[2]
Napište regulární výraz pro nalezení data ve formátu YYYY-MM-DD.
\nopagebreak
\makeemptybox{10cm}
\question[2]
Napište jaký je rozdíl mezi n-ticí a seznamem v Pythonu.
\nopagebreak
\makeemptybox{10cm}
\question[2]
Doplňte tělo funkce fakt(num) v Pythonu 3, taky aby vracela faktoriál celého čísla.
\nopagebreak
\makeemptybox{10cm}
\question[2]
Doplňte tělo funkce divisible(sez) v Pythonu 2 tak, aby vrátil nový seznam čísel, která jsou dělitelná číslem 3.\\
sez je seznam obsahující proměnné typu int
\nopagebreak
\makeemptybox{10cm}
\question[2]
Napište funkci myfilter(array) v jazyce Ruby tak, aby ze vstupního seznamu čísel vytiskla na standardní výstup pouze lichá čísla a čísla dělitelná 10.
\nopagebreak
\makeemptybox{10cm}
\question[2]
Napište program v Pythonu 3, který vypíše na standardní výstup počet nepřekrývajících se řetězců typu YYYY-MM-DD na standardním vstupu.
\nopagebreak
\makeemptybox{10cm}
\question[2]
Napište program v jazyce Python3, Python2 nebo Ruby, který ze standardního vstupu, kterým je CSV soubor obsahující\\
login;počet\_{}bodů\\
login;počet\_{}bodů\\
login;počet\_{}bodů\\
atd...\\
vypíše na standardní výstup celkový počet bodů všech studentů a na nový řádek login studenta s největším počtem bodů.
\nopagebreak
\makeemptybox{10cm}
\question[2]
Regulárnímu výyrazu: \^{}[4-5]\textbackslash\{\}w[\${}\^{}8]\textbackslash\{\}S\textbackslash\{\}d(3\textbar{}6)[\^{}3]\${} odpovídá řetězec:
\nopagebreak
\begin{choices}
\choice
45\^{} 328
\choice
45\${}1938
\choice
55\^{} 936
\choice
40\^{}F726
\choice
56\${} 831
\choice
408 135
\choice
51\${} 122
\choice
418 135
\end{choices}
\question[2]
Napište program v jazyce Ruby který sečte libovolně velké matice a vytiskne výsledek na standardní výstup. Prvky matice jsou odděleny mezerou. Matice na vstupu jsou od sebe odděleny prázdným řádkem a mají stejnou velikost. Příklad vstupu:\\
\texttt{1 1\\
2 2\\
\\
3 3\\
4 4}\\
Odpovídající výstup:\\
\texttt{4 4\\
6 6}\\

\nopagebreak
\makeemptybox{10cm}
\question[2]
Co vypíše následující program v Pythonu 3 na standardní výstup (neuvažujte chybový výstup)?\\
\texttt{def f(n):\\
\hspace*{0.6cm}print('fraz:', n, end=', ')\\
\hspace*{0.6cm}while n \textless{} 8:\\
\hspace*{0.6cm}\hspace*{0.6cm}yield n\\
\hspace*{0.6cm}\hspace*{0.6cm}n += 2\\
\hspace*{0.6cm}\hspace*{0.6cm}print('fdva:', n, end=', ')\\
\hspace*{0.6cm}print('ftri:', n, end=', ')\\
n = 2\\
x = f(n)\\
n += 2\\
print('mraz:', next(x), end=', ')\\
next(x)\\
print('mdva:', n, end=', ')\\
print('mtri:', next(x), end=', ')}
\nopagebreak
\begin{choices}
\choice
fraz: 2, mraz: 2, fdva: 4, mdva: 4, fdva: 6, mtri: 6,
\choice
fraz: 3, mraz: 4, mdva: 4, fdva: 6, mtri: 6,
\choice
fraz: 2, mraz: 3, fdva: 5, mdva: 4, fdva: 7, mtri: 7,
\choice
fraz: 2, mraz: 3, mdva: 4, fdva: 5, mtri: 5,
\choice
fraz: 2, mraz: 2, fdva: 4, fdva: 6, mdva: 6, fdva: 8, ftri: 8,
\choice
fraz: 4, mraz: 5, mdva: 4, fdva: 7, mtri: 7,
\end{choices}
\question[2]
Co vypíše následující program v Pythonu 3 na standardní výstup?\\
\texttt{class Houbogen:\\
\hspace*{0.6cm}def \_{}\_{}init\_{}\_{}(self, n):\\
\hspace*{0.6cm}\hspace*{0.6cm}self.i = n + -4\\
\hspace*{0.6cm}\hspace*{0.6cm}self.j = 3 - n\\
\hspace*{0.6cm}def \_{}\_{}iter\_{}\_{}(self):\\
\hspace*{0.6cm}\hspace*{0.6cm}return self\\
\hspace*{0.6cm}def \_{}\_{}next\_{}\_{}(self):\\
\hspace*{0.6cm}\hspace*{0.6cm}self.j -= 1\\
\hspace*{0.6cm}\hspace*{0.6cm}if self.i \textgreater{}= self.j:\\
\hspace*{0.6cm}\hspace*{0.6cm}\hspace*{0.6cm}raise StopIteration()\\
\hspace*{0.6cm}\hspace*{0.6cm}elif sum((self.i, self.j)) \%{} 3 == 1:\\
\hspace*{0.6cm}\hspace*{0.6cm}\hspace*{0.6cm}self.i += 1\\
\hspace*{0.6cm}\hspace*{0.6cm}\hspace*{0.6cm}print('houbaraz:', self.j, end=', ')\\
\hspace*{0.6cm}\hspace*{0.6cm}\hspace*{0.6cm}return self.i - self.j\\
\hspace*{0.6cm}\hspace*{0.6cm}else:\\
\hspace*{0.6cm}\hspace*{0.6cm}\hspace*{0.6cm}self.i += 1\\
\hspace*{0.6cm}\hspace*{0.6cm}\hspace*{0.6cm}self.j -= 1\\
\hspace*{0.6cm}\hspace*{0.6cm}\hspace*{0.6cm}print('houbadva:', self.i-self.j, end=', ')\\
\hspace*{0.6cm}\hspace*{0.6cm}\hspace*{0.6cm}return self.i - self.j\\
n = 0\\
for h in Houbogen(n):\\
\hspace*{0.6cm}print('les:', h, end=', ')}
\nopagebreak
\begin{choices}
\choice
houbaraz: 2, les: -5, houbaraz: 1, les: -3, houbaraz: 0, les: -1,
\choice
houbadva: 1, les: -3, houbaraz: -1, les: -1,
\choice
houbadva: 1, les: -3, houbaraz: -1, les: -1, houbaraz: 1, les: 1,
\choice
houbaraz: 2, les: -5, houbaraz: 1, les: -3, houbaraz: 0, les: -1, houbaraz: -1, les: 1,
\choice
houbaraz: -5, les: -5, houbaraz: -3, les: -3, houbaraz: -1, les: -1, houbaraz: 1, les: 1,
\choice
houbaraz: -5, les: -5, houbaraz: -3, les: -3, houbaraz: -1, les: -1,
\end{choices}
\question[2]
Co vypíše následující program v Ruby na standardní výstup?\\
\texttt{x = 0\\
z = 2\\
n = 0\\
a = [2,1,4,4]\\
(n...5).each\{\textbar{}y\textbar{} z += x += y\}\\
x, z = z, x\\
x.times \{\textbar{}i\textbar{} z -= 2\}\\
n.upto(a.last) \{\textbar{}y\textbar{} x -= y\}\\
n.upto(a.first+a.first) \{\textbar{}y\textbar{} x -= y\}\\
print x.abs - z.abs}
\nopagebreak
\begin{choices}
\choice
-33
\choice
-42
\choice
-32
\choice
-30
\choice
-18
\choice
-25
\end{choices}
\question[2]
Regulárnímu výyrazu: \^{}((?:38)\{2\}(?:(?:85)\{3\}(62)\{3\})\{3\})\{3\}\textbackslash\{\}2\${} odpovídá řetězec:
\nopagebreak
\begin{choices}
\choice
388562856285623885628562856262
\choice
38858585626262858585626262858585626262388585856262628585856262628585856262623885858562626285858562626285858562626262
\choice
388562626285626262856262623885626262856262628562626262
\choice
38388585856262628585856262628585856262623838858585626262858585626262858585626262383885858562626285858562626285858562626262
\choice
38383885858562626285858562626285858562626238383885858562626285858562626285858562626238383885858562626285858562626285858562626262
\choice
38856262628562626285626262388562626285626262856262623885626262856262628562626262
\end{choices}
\question[2]
Co vypíše následující program v Pythonu 2 na standardní výstup?\\
\texttt{class Houbogen:\\
\hspace*{0.6cm}def \_{}\_{}init\_{}\_{}(self, n):\\
\hspace*{0.6cm}\hspace*{0.6cm}self.i = n + -4\\
\hspace*{0.6cm}\hspace*{0.6cm}self.j = 5 - n\\
\hspace*{0.6cm}def \_{}\_{}iter\_{}\_{}(self):\\
\hspace*{0.6cm}\hspace*{0.6cm}return self\\
\hspace*{0.6cm}def next(self):\\
\hspace*{0.6cm}\hspace*{0.6cm}self.i += 1\\
\hspace*{0.6cm}\hspace*{0.6cm}self.j -= 1\\
\hspace*{0.6cm}\hspace*{0.6cm}if self.i \textgreater{}= self.j:\\
\hspace*{0.6cm}\hspace*{0.6cm}\hspace*{0.6cm}raise StopIteration()\\
\hspace*{0.6cm}\hspace*{0.6cm}elif sum((self.i, self.j)) \%{} 3 == 0:\\
\hspace*{0.6cm}\hspace*{0.6cm}\hspace*{0.6cm}return self.j - self.i\\
\hspace*{0.6cm}\hspace*{0.6cm}else:\\
\hspace*{0.6cm}\hspace*{0.6cm}\hspace*{0.6cm}self.j -= 1\\
\hspace*{0.6cm}\hspace*{0.6cm}\hspace*{0.6cm}return self.j - self.i\\
n = 1\\
print min(Houbogen(n)),}
\nopagebreak
\begin{choices}
\choice
2
\choice
-6
\choice
0
\choice
-4
\choice
-2
\choice
1
\end{choices}
\question[2]
Jak dlouhá je odvěsna pravoúhlého trojúhelníka pokud je druhá odvěsna dlouhá 184 cm a přepona měří 230 cm?
\nopagebreak
\begin{choices}
\choice
139 cm
\choice
141 cm
\choice
138 cm
\choice
142 cm
\choice
137 cm
\choice
140 cm
\end{choices}
\end{questions}
\end{document}

