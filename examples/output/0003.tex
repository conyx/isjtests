\documentclass[addpoints,12pt]{exam}
\usepackage[left=1cm,top=1cm,right=1cm,bottom=2cm,nohead,nofoot]{geometry}
\usepackage[utf8]{inputenc} 
\usepackage[czech]{babel}
\usepackage{amsmath}
\usepackage{amssymb}
\usepackage{stmaryrd}
\begin{document}
\setlength{\parindent}{0in}
\pagestyle{empty}
\textbf{ID: 0003}
\\
\textbf{2. Opravný termín, semestrální test ISJ, 8.6.2012, Login:}
\\
\hrulefill
\begin{questions}

\question[2]
Jaký je rozdíl mezi seznamem (list) a n-ticí (tuple) v Pythonu?
\nopagebreak
\begin{choices}
\choice
Pro n-tici platí, že nelze měnit její velikost ani obsah.
\choice
Na n-tici nelze aplikovat operátor *.
\choice
Seznam není vestavěný typ.
\choice
N-tici nelze iterovat.
\end{choices}
\question[2]
K čemu slouží příkaz puts v jazyk Ruby?
\nopagebreak
\begin{choices}
\choice
Tisk řetězce na výstup.
\choice
Zvýšení velikosti pole o 1.
\choice
Přidání prvku do pole.
\choice
Vyjmutí prvního prvku z pole.
\end{choices}
\question[2]
Který z následujících jazyků nemá vestavěnou podporu pro regulární výrazy?
\nopagebreak
\begin{choices}
\choice
Python 3
\choice
Ruby
\choice
C
\choice
Python 2
\end{choices}
\question[2]
K čemu slouží pass příkaz v Pythonu?
\nopagebreak
\begin{choices}
\choice
Vyhození výjimky.
\choice
Prázdná operace.
\choice
Násilné ukončení cyklu.
\choice
Ukončení programu s návratovou hodnotou 0.
\end{choices}
\question[2]
Který z těchto jazyků je nejvíce objektový?
\nopagebreak
\begin{choices}
\choice
Ruby
\choice
Python 3
\choice
C
\choice
Python 2
\end{choices}
\question[2]
Který z následujících příkazů neexistuje v jazyce Python?
\nopagebreak
\begin{choices}
\choice
puts
\choice
cut
\choice
del
\choice
raise
\end{choices}
\question[2]
Která z následujících tvrzení platí pro datový typ seznam (list) v Pythonu?
\nopagebreak
\begin{choices}
\choice
Můžeme měnit jeho obsah.
\choice
Je takzvaným immutable typem.
\choice
Nepatří do vestavěných typů jazyka Python.
\choice
Můžeme měnit jeho velikost.
\end{choices}
\question[2]
Které řetězce odpovídají následujícímu regulárnímu výrazu [0-9a-fA-F]\{2\}?
\nopagebreak
\begin{choices}
\choice
0L
\choice
16.0
\choice
147
\choice
a9
\end{choices}
\question[2]
Které z následujících jazyků mají zabudovanou podporu (standardní knihovnu) regulárních výrazů?
\nopagebreak
\begin{choices}
\choice
Python
\choice
Java SE
\choice
Ruby
\choice
C
\end{choices}
\question[2]
Která z následujících tvrzení platí pro datový typ n-tice (tuple) v jazyce Python?
\nopagebreak
\begin{choices}
\choice
Je takzvaným immutable typem.
\choice
Můžeme měnit jeho velikost.
\choice
Nepatří do vestavěných typů jazyka Python.
\choice
Můžeme měnit jeho obsah.
\end{choices}
\question[2]
Napište jaký je rozdíl mezi n-ticí a seznamem v Pythonu.
\nopagebreak
\makeemptybox{10cm}
\question[2]
Napište jediným RE přidání oddělení řádu mocnin tečkou (1214248 -\textgreater{} 1.214.248).
\nopagebreak
\makeemptybox{10cm}
\question[2]
Srovnejte jazyky Ruby a Python.
\nopagebreak
\makeemptybox{10cm}
\question[2]
Napište regulární výraz pro nalezení data ve formátu YYYY-MM-DD.
\nopagebreak
\makeemptybox{10cm}
\question[2]
Napište vše co víte o iterátorech v Pythonu.
\nopagebreak
\makeemptybox{10cm}
\question[2]
Napište program v jazyce Python3, Python2 nebo Ruby, který ze standardního vstupu, kterým je CSV soubor obsahující\\
login;počet\_{}bodů\\
login;počet\_{}bodů\\
login;počet\_{}bodů\\
atd...\\
vypíše na standardní výstup celkový počet bodů všech studentů a na nový řádek login studenta s největším počtem bodů.
\nopagebreak
\makeemptybox{10cm}
\question[2]
Doplňte tělo funkce divisible(sez) v Pythonu 2 tak, aby vrátil nový seznam čísel, která jsou dělitelná číslem 3.\\
sez je seznam obsahující proměnné typu int
\nopagebreak
\makeemptybox{10cm}
\question[2]
Napište program v Pythonu 3, který vypíše na standardní výstup počet nepřekrývajících se řetězců typu YYYY-MM-DD na standardním vstupu.
\nopagebreak
\makeemptybox{10cm}
\question[2]
Napište funkci myfilter(array) v jazyce Ruby tak, aby ze vstupního seznamu čísel vytiskla na standardní výstup pouze lichá čísla a čísla dělitelná 10.
\nopagebreak
\makeemptybox{10cm}
\question[2]
Doplňte tělo funkce fakt(num) v Pythonu 3, taky aby vracela faktoriál celého čísla.
\nopagebreak
\makeemptybox{10cm}
\question[2]
Regulárnímu výyrazu: \^{}((66)\{3\}(?:(?:78)\{2\}(81)\{2\})\{2\})\{2\}\textbackslash\{\}3\${} odpovídá řetězec:
\nopagebreak
\begin{choices}
\choice
66666678788166666678788178
\choice
6666667881666666788178
\choice
6666667878818178788181666666787881817878818178788181
\choice
666666787881816666667878818178
\choice
6666667878818178788181666666787881817878818181
\choice
6666667878818178788181666666787881817878818178
\end{choices}
\question[2]
Co vypíše následující program v Ruby na standardní výstup?\\
\texttt{x = 1\\
z = 2\\
n = 1\\
a = [1,4,2,4]\\
(n...5).each\{\textbar{}y\textbar{} z += x += y\}\\
x.times \{\textbar{}i\textbar{} z -= 2\}\\
n.upto(a.length) \{\textbar{}y\textbar{} z -= y\}\\
n.upto(a.first+a.first) \{\textbar{}y\textbar{} x -= y\}\\
print x.abs - z.abs}
\nopagebreak
\begin{choices}
\choice
11
\choice
2
\choice
13
\choice
0
\choice
3
\choice
9
\end{choices}
\question[2]
Jak dlouhá je odvěsna pravoúhlého trojúhelníka pokud je druhá odvěsna dlouhá 108 cm a přepona měří 117 cm?
\nopagebreak
\begin{choices}
\choice
47 cm
\choice
48 cm
\choice
46 cm
\choice
49 cm
\choice
45 cm
\choice
50 cm
\end{choices}
\question[2]
Co vypíše následující program v Pythonu 3 na standardní výstup?\\
\texttt{class Houbogen:\\
\hspace*{0.6cm}def \_{}\_{}init\_{}\_{}(self, n):\\
\hspace*{0.6cm}\hspace*{0.6cm}self.i = n + -4\\
\hspace*{0.6cm}\hspace*{0.6cm}self.j = 4 - n\\
\hspace*{0.6cm}def \_{}\_{}iter\_{}\_{}(self):\\
\hspace*{0.6cm}\hspace*{0.6cm}return self\\
\hspace*{0.6cm}def \_{}\_{}next\_{}\_{}(self):\\
\hspace*{0.6cm}\hspace*{0.6cm}self.j -= 1\\
\hspace*{0.6cm}\hspace*{0.6cm}if self.i \textgreater{} self.j:\\
\hspace*{0.6cm}\hspace*{0.6cm}\hspace*{0.6cm}raise StopIteration()\\
\hspace*{0.6cm}\hspace*{0.6cm}elif sum((self.i, self.j)) \%{} 2 == 1:\\
\hspace*{0.6cm}\hspace*{0.6cm}\hspace*{0.6cm}self.j -= 1\\
\hspace*{0.6cm}\hspace*{0.6cm}\hspace*{0.6cm}print('houbaraz:', self.j, end=', ')\\
\hspace*{0.6cm}\hspace*{0.6cm}\hspace*{0.6cm}return self.j - self.i\\
\hspace*{0.6cm}\hspace*{0.6cm}else:\\
\hspace*{0.6cm}\hspace*{0.6cm}\hspace*{0.6cm}self.j -= 1\\
\hspace*{0.6cm}\hspace*{0.6cm}\hspace*{0.6cm}print('houbadva:', self.i-self.j, end=', ')\\
\hspace*{0.6cm}\hspace*{0.6cm}\hspace*{0.6cm}return self.j - self.i\\
n = 0\\
for h in Houbogen(n):\\
\hspace*{0.6cm}print('les:', h, end=', ')}
\nopagebreak
\begin{choices}
\choice
houbaraz: 2, les: 6, houbaraz: 0, les: 4, houbaraz: -2, les: 2, houbaraz: -4, les: 0,
\choice
houbadva: -5, les: 5, houbadva: -3, les: 3, houbadva: -1, les: 1, houbadva: 1, les: -1,
\choice
houbaraz: 2, les: 5, houbadva: -3, les: 3, houbadva: -1, les: 1, houbadva: 1, les: -1,
\choice
houbadva: 5, les: 5, houbadva: 3, les: 3, houbadva: 1, les: 1, houbadva: -1, les: -1,
\choice
houbadva: 5, les: 5, houbaraz: -1, les: -2, houbadva: 0, les: 0,
\choice
houbaraz: 1, les: -3, houbadva: 1, les: 1, houbadva: -1, les: -1,
\end{choices}
\question[2]
Co vypíše následující program v Pythonu 2 na standardní výstup?\\
\texttt{class Houbogen:\\
\hspace*{0.6cm}def \_{}\_{}init\_{}\_{}(self, n):\\
\hspace*{0.6cm}\hspace*{0.6cm}self.i = n + -3\\
\hspace*{0.6cm}\hspace*{0.6cm}self.j = 4 - n\\
\hspace*{0.6cm}def \_{}\_{}iter\_{}\_{}(self):\\
\hspace*{0.6cm}\hspace*{0.6cm}return self\\
\hspace*{0.6cm}def next(self):\\
\hspace*{0.6cm}\hspace*{0.6cm}self.i += 1\\
\hspace*{0.6cm}\hspace*{0.6cm}if self.i \textgreater{} self.j:\\
\hspace*{0.6cm}\hspace*{0.6cm}\hspace*{0.6cm}raise StopIteration()\\
\hspace*{0.6cm}\hspace*{0.6cm}elif sum((self.i, self.j)) \%{} 2 == 1:\\
\hspace*{0.6cm}\hspace*{0.6cm}\hspace*{0.6cm}self.i += 1\\
\hspace*{0.6cm}\hspace*{0.6cm}\hspace*{0.6cm}self.j -= 1\\
\hspace*{0.6cm}\hspace*{0.6cm}\hspace*{0.6cm}return self.i - self.j\\
\hspace*{0.6cm}\hspace*{0.6cm}else:\\
\hspace*{0.6cm}\hspace*{0.6cm}\hspace*{0.6cm}self.i += 1\\
\hspace*{0.6cm}\hspace*{0.6cm}\hspace*{0.6cm}return self.i - self.j\\
n = 1\\
print sum(Houbogen(n))}
\nopagebreak
\begin{choices}
\choice
3
\choice
2
\choice
-8
\choice
-6
\choice
-3
\choice
-10
\end{choices}
\question[2]
Napište program v jazyce Python 3 který sečte libovolně velké matice a vytiskne výsledek na standardní výstup. Prvky matice jsou odděleny mezerou. Matice na vstupu jsou od sebe odděleny prázdným řádkem a mají stejnou velikost. Příklad vstupu:\\
\texttt{1 1\\
2 2\\
\\
3 3\\
4 4}\\
Odpovídající výstup:\\
\texttt{4 4\\
6 6}\\

\nopagebreak
\makeemptybox{10cm}
\question[2]
Co vypíše následující program v Pythonu 3 na standardní výstup (neuvažujte chybový výstup)?\\
\texttt{def f(n):\\
\hspace*{0.6cm}print('fraz:', n, end=', ')\\
\hspace*{0.6cm}n += 2\\
\hspace*{0.6cm}while n \textless{} 8:\\
\hspace*{0.6cm}\hspace*{0.6cm}yield n\\
\hspace*{0.6cm}\hspace*{0.6cm}n += 1\\
\hspace*{0.6cm}\hspace*{0.6cm}print('fdva:', n, end=', ')\\
\hspace*{0.6cm}print('ftri:', n, end=', ')\\
n = 0\\
x = f(n)\\
print('mraz:', n, end=', ')\\
print('mdva:', n, end=', ')\\
next(x)\\
print('mtri:', next(x), end=', ')}
\nopagebreak
\begin{choices}
\choice
mraz: 0, mdva: 0, fraz: 0, fdva: 3, mtri: 3,
\choice
mraz: 0, fraz: 0, mdva: 2, fdva: 3, fdva: 4, mtri: 4,
\choice
mraz: 0, fraz: 0, mdva: 2, mtri: 0,
\choice
fraz: 0, mraz: 2, fdva: 4, mdva: 4, fdva: 6, mtri: 0,
\choice
mraz: 0, fraz: 0, mdva: 2, fdva: 3, mtri: 0,
\choice
mraz: 0, fraz: 0, mdva: 2, fdva: 4, mtri: 0,
\end{choices}
\question[2]
Regulárnímu výyrazu: \^{}[\^{}2]\textbackslash\{\}s\textbackslash\{\}d\textbackslash\{\}w[\^{}9][1-8](19)\${} odpovídá řetězec:
\nopagebreak
\begin{choices}
\choice
. 4\${}4619
\choice
. 5\${}7419
\choice
2 5\${}1519
\choice
9 155819
\choice
\^{} 5\${}551
\choice
\^{} 4\${}441
\choice
\^{} 6\${}7419
\choice
\^{} 929419
\end{choices}
\end{questions}
\end{document}

