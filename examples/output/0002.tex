\documentclass[addpoints,12pt]{exam}
\usepackage[left=1cm,top=1cm,right=1cm,bottom=2cm,nohead,nofoot]{geometry}
\usepackage[utf8]{inputenc} 
\usepackage[czech]{babel}
\usepackage{amsmath}
\usepackage{amssymb}
\usepackage{stmaryrd}
\begin{document}
\setlength{\parindent}{0in}
\pagestyle{empty}
\textbf{ID: 0002}
\\
\textbf{2. Opravný termín, semestrální test ISJ, 8.6.2012, Login:}
\\
\hrulefill
\begin{questions}

\question[2]
K čemu slouží příkaz puts v jazyk Ruby?
\nopagebreak
\begin{choices}
\choice
Tisk řetězce na výstup.
\choice
Přidání prvku do pole.
\choice
Vyjmutí prvního prvku z pole.
\choice
Zvýšení velikosti pole o 1.
\end{choices}
\question[2]
Jaký je rozdíl mezi seznamem (list) a n-ticí (tuple) v Pythonu?
\nopagebreak
\begin{choices}
\choice
Na n-tici nelze aplikovat operátor *.
\choice
Pro n-tici platí, že nelze měnit její velikost ani obsah.
\choice
Seznam není vestavěný typ.
\choice
N-tici nelze iterovat.
\end{choices}
\question[2]
K čemu slouží pass příkaz v Pythonu?
\nopagebreak
\begin{choices}
\choice
Vyhození výjimky.
\choice
Násilné ukončení cyklu.
\choice
Prázdná operace.
\choice
Ukončení programu s návratovou hodnotou 0.
\end{choices}
\question[2]
Který z těchto jazyků je nejvíce objektový?
\nopagebreak
\begin{choices}
\choice
C
\choice
Python 3
\choice
Ruby
\choice
Python 2
\end{choices}
\question[2]
Který z následujících jazyků nemá vestavěnou podporu pro regulární výrazy?
\nopagebreak
\begin{choices}
\choice
Ruby
\choice
Python 2
\choice
Python 3
\choice
C
\end{choices}
\question[2]
Který z následujících příkazů neexistuje v jazyce Python?
\nopagebreak
\begin{choices}
\choice
cut
\choice
raise
\choice
del
\choice
puts
\end{choices}
\question[2]
Které řetězce odpovídají následujícímu regulárnímu výrazu [0-9a-fA-F]\{2\}?
\nopagebreak
\begin{choices}
\choice
a9
\choice
16.0
\choice
0L
\choice
147
\end{choices}
\question[2]
Která z následujících tvrzení platí pro datový typ n-tice (tuple) v jazyce Python?
\nopagebreak
\begin{choices}
\choice
Je takzvaným immutable typem.
\choice
Nepatří do vestavěných typů jazyka Python.
\choice
Můžeme měnit jeho velikost.
\choice
Můžeme měnit jeho obsah.
\end{choices}
\question[2]
Která z následujících tvrzení platí pro datový typ seznam (list) v Pythonu?
\nopagebreak
\begin{choices}
\choice
Můžeme měnit jeho obsah.
\choice
Je takzvaným immutable typem.
\choice
Nepatří do vestavěných typů jazyka Python.
\choice
Můžeme měnit jeho velikost.
\end{choices}
\question[2]
Které z následujících jazyků mají zabudovanou podporu (standardní knihovnu) regulárních výrazů?
\nopagebreak
\begin{choices}
\choice
C
\choice
Ruby
\choice
Java SE
\choice
Python
\end{choices}
\question[2]
Napište jediným RE přidání oddělení řádu mocnin tečkou (1214248 -\textgreater{} 1.214.248).
\nopagebreak
\makeemptybox{10cm}
\question[2]
Napište jaký je rozdíl mezi n-ticí a seznamem v Pythonu.
\nopagebreak
\makeemptybox{10cm}
\question[2]
Srovnejte jazyky Ruby a Python.
\nopagebreak
\makeemptybox{10cm}
\question[2]
Napište vše co víte o iterátorech v Pythonu.
\nopagebreak
\makeemptybox{10cm}
\question[2]
Napište regulární výraz pro nalezení data ve formátu YYYY-MM-DD.
\nopagebreak
\makeemptybox{10cm}
\question[2]
Napište program v jazyce Python3, Python2 nebo Ruby, který ze standardního vstupu, kterým je CSV soubor obsahující\\
login;počet\_{}bodů\\
login;počet\_{}bodů\\
login;počet\_{}bodů\\
atd...\\
vypíše na standardní výstup celkový počet bodů všech studentů a na nový řádek login studenta s největším počtem bodů.
\nopagebreak
\makeemptybox{10cm}
\question[2]
Napište funkci myfilter(array) v jazyce Ruby tak, aby ze vstupního seznamu čísel vytiskla na standardní výstup pouze lichá čísla a čísla dělitelná 10.
\nopagebreak
\makeemptybox{10cm}
\question[2]
Napište program v Pythonu 3, který vypíše na standardní výstup počet nepřekrývajících se řetězců typu YYYY-MM-DD na standardním vstupu.
\nopagebreak
\makeemptybox{10cm}
\question[2]
Doplňte tělo funkce fakt(num) v Pythonu 3, taky aby vracela faktoriál celého čísla.
\nopagebreak
\makeemptybox{10cm}
\question[2]
Doplňte tělo funkce divisible(sez) v Pythonu 2 tak, aby vrátil nový seznam čísel, která jsou dělitelná číslem 3.\\
sez je seznam obsahující proměnné typu int
\nopagebreak
\makeemptybox{10cm}
\question[2]
Co vypíše následující program v Ruby na standardní výstup?\\
\texttt{x = 1\\
z = 2\\
n = 2\\
a = [1,2,3,4]\\
(n..7).each\{\textbar{}y\textbar{} z += x += y\}\\
x.times \{\textbar{}i\textbar{} z -= 1\}\\
n.upto(a.length) \{\textbar{}y\textbar{} z -= y\}\\
n.upto(a.first+a.first) \{\textbar{}y\textbar{} x -= y\}\\
print x.abs - z.abs}
\nopagebreak
\begin{choices}
\choice
-8
\choice
-22
\choice
-13
\choice
-14
\choice
-17
\choice
-18
\end{choices}
\question[2]
Regulárnímu výyrazu: \^{}\textbackslash\{\}W[\^{}4][\^{}5][3-5](7\textbar{}5)\textbackslash\{\}S\textbackslash\{\}d\${} odpovídá řetězec:
\nopagebreak
\begin{choices}
\choice
\${}1171 6
\choice
\${}5537a1
\choice
\${}2397 6
\choice
\${}1675 4
\choice
\${}3935 6
\choice
\${}4735A9
\choice
\${}985536
\choice
\${}6637 8
\end{choices}
\question[2]
Co vypíše následující program v Pythonu 2 na standardní výstup?\\
\texttt{class Houbogen:\\
\hspace*{0.6cm}def \_{}\_{}init\_{}\_{}(self, n):\\
\hspace*{0.6cm}\hspace*{0.6cm}self.i = n + -3\\
\hspace*{0.6cm}\hspace*{0.6cm}self.j = 5 - n\\
\hspace*{0.6cm}def \_{}\_{}iter\_{}\_{}(self):\\
\hspace*{0.6cm}\hspace*{0.6cm}return self\\
\hspace*{0.6cm}def next(self):\\
\hspace*{0.6cm}\hspace*{0.6cm}self.i += 1\\
\hspace*{0.6cm}\hspace*{0.6cm}if self.i \textgreater{} self.j:\\
\hspace*{0.6cm}\hspace*{0.6cm}\hspace*{0.6cm}raise StopIteration()\\
\hspace*{0.6cm}\hspace*{0.6cm}elif sum((self.i, self.j)) \%{} 3 == 0:\\
\hspace*{0.6cm}\hspace*{0.6cm}\hspace*{0.6cm}self.i += 1\\
\hspace*{0.6cm}\hspace*{0.6cm}\hspace*{0.6cm}self.j -= 1\\
\hspace*{0.6cm}\hspace*{0.6cm}\hspace*{0.6cm}return self.i - self.j\\
\hspace*{0.6cm}\hspace*{0.6cm}else:\\
\hspace*{0.6cm}\hspace*{0.6cm}\hspace*{0.6cm}return self.i - self.j\\
n = 1\\
print min(Houbogen(n)),}
\nopagebreak
\begin{choices}
\choice
-3
\choice
3
\choice
4
\choice
0
\choice
-1
\choice
1
\end{choices}
\question[2]
Napište program v jazyce Python 2 který sečte libovolně velké matice a vytiskne výsledek na standardní výstup. Prvky matice jsou odděleny mezerou. Matice na vstupu jsou od sebe odděleny prázdným řádkem a mají stejnou velikost. Příklad vstupu:\\
\texttt{1 1\\
2 2\\
\\
3 3\\
4 4}\\
Odpovídající výstup:\\
\texttt{4 4\\
6 6}\\

\nopagebreak
\makeemptybox{10cm}
\question[2]
Jak dlouhá je odvěsna pravoúhlého trojúhelníka pokud je druhá odvěsna dlouhá 273 cm a přepona měří 327 cm?
\nopagebreak
\begin{choices}
\choice
182 cm
\choice
179 cm
\choice
178 cm
\choice
180 cm
\choice
181 cm
\choice
177 cm
\end{choices}
\question[2]
Regulárnímu výyrazu: \^{}((27)\{3\}((64)\{3\}(84)\{2\})\{3\})\{1\}\textbackslash\{\}5\${} odpovídá řetězec:
\nopagebreak
\begin{choices}
\choice
2764648484646484846464848484
\choice
2764648484646484846464848464
\choice
27272764648484646484846464848464
\choice
27272764646484846464648484646464848484
\choice
27272764648484646484846464848484
\choice
272727646464848484
\end{choices}
\question[2]
Co vypíše následující program v Pythonu 3 na standardní výstup?\\
\texttt{class Houbogen:\\
\hspace*{0.6cm}def \_{}\_{}init\_{}\_{}(self, n):\\
\hspace*{0.6cm}\hspace*{0.6cm}self.i = n + -5\\
\hspace*{0.6cm}\hspace*{0.6cm}self.j = 3 - n\\
\hspace*{0.6cm}def \_{}\_{}iter\_{}\_{}(self):\\
\hspace*{0.6cm}\hspace*{0.6cm}return self\\
\hspace*{0.6cm}def \_{}\_{}next\_{}\_{}(self):\\
\hspace*{0.6cm}\hspace*{0.6cm}self.i += 1\\
\hspace*{0.6cm}\hspace*{0.6cm}if self.i \textgreater{} self.j:\\
\hspace*{0.6cm}\hspace*{0.6cm}\hspace*{0.6cm}raise StopIteration()\\
\hspace*{0.6cm}\hspace*{0.6cm}elif sum((self.i, self.j)) \%{} 2 == 1:\\
\hspace*{0.6cm}\hspace*{0.6cm}\hspace*{0.6cm}self.i += 1\\
\hspace*{0.6cm}\hspace*{0.6cm}\hspace*{0.6cm}self.j -= 1\\
\hspace*{0.6cm}\hspace*{0.6cm}\hspace*{0.6cm}print('houbaraz:', self.i, end=', ')\\
\hspace*{0.6cm}\hspace*{0.6cm}\hspace*{0.6cm}return self.i - self.j\\
\hspace*{0.6cm}\hspace*{0.6cm}else:\\
\hspace*{0.6cm}\hspace*{0.6cm}\hspace*{0.6cm}print('houbadva:', self.i-self.j, end=', ')\\
\hspace*{0.6cm}\hspace*{0.6cm}\hspace*{0.6cm}return self.i - self.j\\
n = 1\\
for h in Houbogen(n):\\
\hspace*{0.6cm}print('les:', h, end=', ')}
\nopagebreak
\begin{choices}
\choice
houbaraz: -4, les: -5, houbadva: 4, les: -4, houbaraz: -3, les: -1,
\choice
houbaraz: -3, les: -5, houbadva: 4, les: -4, houbaraz: 0, les: -1,
\choice
houbaraz: -2, les: -3, houbadva: 2, les: -2, houbaraz: 1, les: 1,
\choice
houbaraz: -3, les: -5, houbadva: 4, les: 4, houbaraz: 0, les: -1,
\choice
houbaraz: -2, les: -3, houbadva: 2, les: 2, houbaraz: 1, les: 1,
\choice
houbaraz: -2, les: -3, houbadva: -2, les: -2, houbaraz: 1, les: 1,
\end{choices}
\question[2]
Co vypíše následující program v Pythonu 3 na standardní výstup (neuvažujte chybový výstup)?\\
\texttt{def f(n):\\
\hspace*{0.6cm}n = 2\\
\hspace*{0.6cm}print('fraz:', n, end=', ')\\
\hspace*{0.6cm}n += 2\\
\hspace*{0.6cm}while n \textless{} 6:\\
\hspace*{0.6cm}\hspace*{0.6cm}yield n\\
\hspace*{0.6cm}\hspace*{0.6cm}n += 1\\
\hspace*{0.6cm}\hspace*{0.6cm}print('fdva:', n, end=', ')\\
\hspace*{0.6cm}print('ftri:', n, end=', ')\\
n = 2\\
x = f(n)\\
n += 2\\
x = f(n)\\
print('mraz:', n, end=', ')\\
next(x)\\
print('mdva:', n, end=', ')\\
next(x)\\
print('mtri:', next(x), end=', ')}
\nopagebreak
\begin{choices}
\choice
mraz: 0, fraz: 2, mdva: 0, fdva: 3, fdva: 4, mtri: 4,
\choice
mraz: 0, fraz: 2, mdva: 0, fdva: 5, fdva: 6, ftri: 6,
\choice
mraz: 2, fraz: 2, mdva: 2, fdva: 5, fdva: 6, ftri: 6,
\choice
mraz: 0, mdva: 0, fraz: 2, fdva: 3, mtri: 3,
\choice
mraz: 4, fraz: 2, mdva: 4, fdva: 5, fdva: 6, ftri: 6,
\choice
mraz: 0, fraz: 2, mdva: 0, fdva: 4, fdva: 6, mtri: 6,
\end{choices}
\end{questions}
\end{document}

